\documentclass[
12pt, % Main document font size
a4paper, % Paper type, use 'letterpaper' for US Letter paper
oneside, % One page layout (no page indentation)
]{scrartcl}
\usepackage[german]{babel}
\title{Steigt der Hamburger SV ab?}
\author{Sandra Dylus\\Institut f\"ur Informatik, CAU Kiel, D-24098 Kiel\\\texttt{sad@informatik.uni-kiel.de}}
\date{}
\begin{document}
\maketitle

Viele Sportturniere unterliegen einem Punktesystem. %
In diesem Vortrag soll das Punktesystem und daraus entstehende
Fragestellungen der 1. Fu\ss{}ball Bundesliga genauer beleuchtet
werden. %
Es handelt sich dabei um ein Turnier, bei dem jede Mannschaft der Liga
genau zwei mal gegen jede andere Mannschaft der Liga spielt. %
Das h\"ochste Ziel dieses Turnier ist es, Meister zu werden und somit
mehr Punkte zu erzielen als alle anderen Mannschaften. %
Die zwei Mannschaften mit den wenigsten Punkten d\"urfen leider nicht
in der 1. Fu\ss{}ball Bundesliga bleiben und steigen ab. %

Ingo Wegener ist vor etwa 15 Jahren der Frage nachgegangen, ob sein
Lieblingsverein Werder Bremen bei einer gegeben Ausgangslange noch
Meister werden kann. %
Im Zusammenhang eines Seminars und zwei folgenden Diplomarbeiten hat
er mit seinen Studenten erarbeitet, dass diese Frage im aktuellen
Punktesystem der Liga sogar NP-vollst\"andig~\cite{threePointRule} ist. %

Leider ist das sogenannte Meisterschaftsproblem aus meiner Sicht
utopisch; vielmehr ist es f\"ur mich relevant, ob der Hamburger SV
nicht absteigt und somit in der 1. Fu\ss{}ball Bundesliga verbleiben
kann. %
Das prinzipielle Vorgehen bei der Probleml\"osung bleibt jedoch
gleich. %

W\"ahrend die Schlagzeilen aktuell eher gegen den Hamburger SV
sprechen, m\"ochte ich es aber ganz genau wissen. %
Kann der Hamburger SV zum Zeitpunkt des Workshops noch auf den
Verbleib in der 1. Fu\ss{}ball Bundesliga hoffen? %
Ist es auch m\"oglich, die Wahrscheinlichkeit zu berechnen,
mit der der Hamburger SV in der Liga bleibt? %
Zun\"achst werden das vorliegende Punktesystem und relevante Regeln des
Turniers erarbeitet und mit Hilfe der funktional logischen
Programmiersprache Curry soll dann berechnet werden, ob der gew\"unschte
Ausgang f\"ur den Hamburger SV noch m\"oglich ist. %

\bibliography{/Users/sad/Documents/Papers/used}
\bibliographystyle{plain}
\end{document}